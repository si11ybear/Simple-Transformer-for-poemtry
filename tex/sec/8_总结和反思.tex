\section{总结与反思}\label{sec-8}

\paragraph{丁玺博} 我在本项目中主要负责两篇文献阅读和模型的落地实现部分,整体来说基本完成全部目标。这个过程并不顺利,主要困难是集中在模型的封装调用和改进文本生成逻辑两部分。我清楚认识到一个模型从基本的训练完成到可以自由使用,再到根据实际需求针对性改造与使用,实际上是两个非常艰难、漫长和复杂的过程。特别是结合现实需求的设计,我需要不断去考虑一切可能影响到文本生成的因素,比如模型词向量嵌入步长,src和tgt的动态变化,推理模式随机性的控制等等,并尝试与实际生成古诗文的需求结合起来,比如平仄和韵律。这实在是对一个人耐心和逻辑的极大挑战。如果简单将模型的训练和生成作为后端,用户使用作为前端,那么我确实是对前后端融合以及前端的工作有了更深的认识,这些都是之前仅仅关注模型训练的我所不曾掌握的知识。未来的学习和工作中,我也会更多考虑现实需求和框架融合,而不仅仅关注模型的进步。这学期学习到了非常多有关深度学习的知识,并在手动执行后有了更多的体悟。感谢老师和助教的辛苦付出,遥祝新年快乐,一切顺利。

\paragraph{李健宁} 我在本项目中主要负责精度Transformer原始论文,并仅利用Pytorch基础模块和线性层搭建了自己的Transformer架构。由于之前几乎没有使用Pytorch的经验,在刚开始接触这一部分时第六次作业带给我很大启发。从一开始不清楚初始化和前向传播函数,到能自己完全写出一个模块,在编程过程中进一步感受到了类在编程中的重要作用。在一步一步实现Transformer的过程中,不仅对Transformer的细节有了更加深刻的理解,也对Pytorch中的并行算子。批处理、损失函数等有了更为详细的认知。

代码和算法能力只有在实操中才能得到飞跃,在未来的学习中,我会尝试更多的利用机器学习和深度学习,注重模型的细节和不同模型间的联系,将理论所学与实际操作相结合起来。最后,感谢老师和助教的辛苦付出,让我能深切感受到深度学习的魅力和价值。

\paragraph{肖庆成} 在本项目中,我的主要职责涵盖了两篇关键文献的精读、现有工作的调研、数据集预处理以及将创新性改进应用于自建的Transformer模型。尽管整个过程复杂多变,但总体进展顺利。其中最具有挑战性的环节是将定制化的改进措施融入到手动实现的Transformer架构中,这要求我克服与Python库内置Transformer模型之间的变量命名和函数差异的问题。经过细致的对比和调整,最终成功实现了这一目标。

对于未来的工作,我认为可以在数据集预处理阶段投入更多精力,以期通过引入更丰富的数据资源来进一步优化模型性能。此外,我还意识到,在今后的学习和职业发展中,应更加注重实际应用场景的需求以及不同技术框架间的兼容性,而不仅仅是追求模型算法上的突破。这个学期里,我不仅深入学习了深度学习领域的广泛知识,而且通过实践操作加深了对这些理论的理解。感谢老师和助教团队在整个学期中的悉心指导和支持,祝您们新年快乐,万事如意!

